\documentclass[11pt]{article}
\usepackage[a4paper]{geometry}
\usepackage{polski}
\usepackage{hyperref}
\usepackage[utf8]{inputenc}
\usepackage[table,xcdraw]{xcolor}
\usepackage{graphicx}[demo]
\usepackage{tikz}
\usepackage{float}
\usepackage[usestackEOL]{stackengine} 
\usepackage{caption}
\usetikzlibrary{shapes,arrows,chains}
\usetikzlibrary[calc]
\linespread{1.3}
\usepackage{listings}
\usepackage{indentfirst}

\begin{document}
\begin{titlepage}
\newcommand{\HRule}{\rule{\linewidth}{0.5mm}} % Defines a new command for the horizontal lines, change thickness here
\center % Center everything on the page
%	LOGO SECTION
\includegraphics[scale = 0.21]{pwr-logo.png}\\[2cm]
%	HEADING SECTIONS
\textsc{\Large Programowanie Obiektowe - Labolatorium}\\[0.5cm] 
\textsc{\large projekt wtorek 18:55}\\[0.5cm]
%	TITLE SECTION
\HRule \\[0.4cm]
{ \huge \bfseries Fantasy Battle Simulator
}\\[0.4cm] 
\HRule \\[0.8cm]
%	AUTHORS SECTION
\begin{minipage}{0.5\textwidth}
\begin{flushleft} \large
\emph{Autor:}\\
Łukasz \textsc{Czarniecki} 272922  Leader\\
Alekasnder \textsc{Torenc} 281184 \\
Franciszek \textsc{Pawlik} 281181 \\

\end{flushleft}
\end{minipage}
~
\begin{minipage}{0.4\textwidth}
\begin{flushright} \large
\emph{Prowadzący:} \\ 
mgr inż. Tobiasz \textsc{Puślecki} % supervisor
\end{flushright}
\end{minipage}\\[5cm]

\vfill % Fill the rest of the page with whitespace
\end{titlepage}
\newgeometry{bmargin=2cm, tmargin=2cm, lmargin=2cm, rmargin=2cm}
\newpage




\section{Wstęp}

Celem projektu jest stworzenie autonomicznego systemu walki 2 drużyn, gdzie jednostki podejmują decyzje w oparciu o własną wiedzię i zamiary. Użytkownik tworzy drużynę startową, walka toczy się bez jego ingerencji.

\section{Opis}

Użytkownik przygotowuje maksymalnie 4-osobową drużynę w oparciu o rasy, klasy, ekwipunek i umiejętności. Po rozpoczęciu walki postacie same wykonują tury na zmianę do zwycięstwa jednej drużyny. W trakcie walki możemy obserwować pojedyńcze akcje jednostek.

%Cytat \cite{different}.

%\[ \sum_{i=1}^{N} B_i X_i \]

%\begin{itemize}
%    \item $B_i$ - wartość i-tego przedmiotu,
%\end{itemize}



\section{Wyniki pomiarów}

\subsection{Algorytm}

%Co przekłada się na błąd względny optimum:

%\noindent\begin{minipage}{\linewidth}
%  \centering
%        \begin{tabular}{|c | c|c |c|}
%            \hline
%            \rowcolor{lightgray}
%            Instancja & Wynik  & Optimum & Błąd względny\\
%            \hline  
%            knapPI-1 & 64691 & 563647 & 11.5\%\\
%            knapPI-2 & 51886 & 90204 & 57.5\%\\
%            knapPI-3 & 61368 & 146919& 41.8\%\\
%
%            \hline  
%        \end{tabular}
%  \captionof{table}{Błąd względny względem optimum}
%  \label{tab:test1}
%\end{minipage}

%DIAGRAM - - - - - - - - - - - -- - - - - - -- - - -- - -- - -- - -- - -- 

\subsection{Diagramy}

\begin{figure}[h!] 
\includegraphics[scale=0.5]{Character hierarchy.png}
\caption{Diagram klas postaci}
\end{figure}

\begin{figure}[h!] 
\includegraphics[scale=0.5]{Item-Character Relation.png}
\caption{Diagram przedmiotów}
\end{figure}



\section{Wnioski}

Dobra gra, nieźle rozegrana.


\begin{thebibliography}{9}
%\bibitem{different}
%Maya Hristakeva - Different Approaches to Solve the 0/1 Knapsack Problem, Simpson College.

\end{thebibliography}

\end{document}